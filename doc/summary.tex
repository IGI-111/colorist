\documentclass[10pt,a4paper]{article}

\usepackage[a4paper]{geometry}
\usepackage[utf8]{inputenc}
\usepackage{fancyhdr}
\usepackage{titlesec}
\usepackage{graphicx}
\usepackage[backend=biber]{biblatex}
\addbibresource{biblio.bib}


\title{\textbf{TP Algorithmique \& Programmation} \\ Coloriage d'images}
\author{Vincent Bonnecuelle et Jibril Saffi}
\date{Octobre 2015}

\pagestyle{fancy}
\fancyhead[R]{1AA}
\fancyhead[L]{Grenoble INP \-- Ensimag}

\renewcommand\thesubsection{\arabic{subsection}}

\titleformat{\subsubsection}
{\normalfont\large\bfseries}{Question \thesubsubsection}{1em}{}

\titleformat{\subsection}[runin]
{\normalfont\large\bfseries}{\thesubsection}{1em}{}

\titleformat{\section}[runin]
{\normalfont\large\bfseries}{}{1em}{}

\begin{document}
\maketitle
\hrule
\section*{Réponses aux questions}
\subsection{Lecture/écriture de données}
\subsubsection{} 
Nous avons identifié les cas d'erreurs suivants pour la fonction \texttt{readMonochrome()} :
\begin{itemize}
\item le fichier n'est pas du bon type
\item il manque des valeurs dans le fichier par rapport au header spécifié
\item la forme du header n'est pas valide 
\item $n \cdot m$ ne renvoie pas un entier positif ou nul.
\item un commentaire est placé après le "nombre magique"
\end{itemize}
Les tests pour ces erreurs sont présents dans le fichier \textit{src/test/pixmap.cxx}.

\subsubsection{} 
Nous avons identifié les cas d'erreurs suivants pour la fonction \texttt{writeColored()} :
\begin{itemize}
\item le tableau bidimensionnel de pixels de couleur est vide
\item le tableau bidimensionnel de pixels de couleur n'est pas rectangulaire : il faut que les lignes aient la même longueur.
\end{itemize}
Les tests pour ces erreurs sont présents dans le fichier \textit{src/test/pixmap.cxx}.

\subsubsection{}
L'implémentation de la fonction \texttt{randomMonochrome()} se trouve dans le fichier \textit{src/pixmap.cxx}. Un paramètre a été ajouté pour permettre à l'utilisateur de sélectionner une méthode de distribution aléatoire.

\clearpage
\subsection{\texttt{MakeSet()} et \texttt{FindSet()}}

\subsubsection{}
Les tests suivants ont été réalisés pour les fonctions {\texttt{MakeSet()} et \texttt{FindSet()} :
\begin{itemize}
\item Initialisation avec une coordonnée incohérente
\item Création d'un ensemble et vérification de l'existence de cet ensemble avec FindSet()
\end{itemize}

\subsubsection{} \label{2.2}
La complexité de \texttt{MakeSet()} est égale à celle de \texttt{FindSet()} soit $\Theta(1)$.

\subsection{Union()}
\subsubsection{} \label{3.1}
Soit 2 ensembles disjoints $S_{1}$ et $S_{2}$ représentés par les listes chaînées $L_{1}$ et $L_{2}$. Afin de réaliser l'union de ces 2 ensembles, la solution naïve consiste à joindre la tête de $L_{2}$ à la queue de $L_{1}$ et de mettre à jour tous les éléments (dont le représentant) de $L_{2}$ afin que ces derniers pointent sur le représentant de $L_{1}$.

\begin{figure}[h]
\centering
\includegraphics{3-1.pdf}
\caption{Mécanisme de \texttt{Union()}}
\end{figure}

Si l'on considère que le pire cas correspond au cas où chaque union ferait systématiquement parcourir l'ensemble le plus grand (pour mettre à jour les éléments). Alors pour $n$ unions on doit faire $n(n-1)?$ opérations\footnote{avec $k? = \sum\limits_{i=1}^{k}{i}$} or:
$$O(n(n-1)?) =  O(n(2(n-1))) = O(n^2)$$

Concernant le coût mémoire, aucunes variables supplémentaires n'est nécessaire car on se contente de modifier la structure d'une liste préalablement existante. Ainsi, le coût en mémoire serait de $O(1)$.
\subsubsection{}

la création d'un ensemble est de complexité $O(1)$ donc $n$ créations d'ensembles aura pour complexité $O(n)$. Succédée de $n$ unions successives, la complexité atteindrais donc :
$$O(n + (n-1)^2) = O(n^2)$$

\subsubsection{}
Dans la question~\ref{3.1} nous avons dégagé une problématique intéressante : si la taille de $L_{2}$ est supérieure à celle de $L_{1}$ nous sommes tout de même contraint de parcourir entièrement la liste $L_{2}$ pour pouvoir joindre celle-ci à $L_{1}$. Aussi, nous avons effectués une modification à la structure de données de liste chaînée : l'\textbf{ajout d'un champ représentant la taille de la liste}.

Grâce à cette information supplémentaire, il nous est désormais possible d'optimiser l'union de deux ensembles. Pour ce faire, on compare la taille des deux listes afin de systématiquement \textbf{joindre la tête de la liste la plus petite vers la queue de la liste la plus grande} et ainsi minimiser le temps de parcours nécessaire à la mise à jour des éléments.

\subsubsection{}
\textit{Voir fichier src/disjoint.hxx}

\subsubsection{}\label{3.5}
L'union implique nécessairement deux ensembles disjoints, nous réalisons donc au minimum $n-1$ opérations d'unions. Considérant un élément $x$ de la liste, nous savons que chaque fois que $x$ est mis à jour, il doit être contenu dans l'ensemble le plus petit. Aussi, la première fois que $x$ est mis à jour, l'ensemble résultant doit contenir au moins 2 éléments. De la même manière, la prochaine fois que $x$ est mis à jour, l'ensemble résultant devra quand à lui contenir au moins 4 éléments. En continuant ainsi, on remarque comme vu en figure~\ref{fig:tree} que pour tout $k \leq n$, si $x$ a été mis à jour $\log(k)$ fois, l'ensemble résultant contient alors $k$ éléments.

\begin{figure}
\centering
\includegraphics{3-5.pdf}
\caption{Les \texttt{Union()} décrivent un arbre binaire de profondeur $\log(n)$}
\label{fig:tree}
\end{figure}


Or l'ensemble le plus grand contient au maximum $n$ éléments, donc chacun de ces éléments est au moins mis à jour $\log(n)$ fois lors des opérations d'unions. Ainsi le temps total alloué à la mise à jour des éléments lors de toutes les opérations d'unions est : 
$$O(n\log(n))$$
De plus comme montré en~\ref{2.2} l'opération \texttt{MakeSet()} a une complexité $O(1)$ et on fait $n$ opérations.
La complexité totale de la suite d'opérations est donc:
$$O(n+n\log(n)) = O(n\log(n)$$

\subsection{Algorithme général}
\subsubsection{}
\subsubsection{}
Soit $n$ et $m$ respectivement la longueur et largeur de l'image.
Comme vu en~\ref{2.2}, le coût en mémoire de \texttt{MakeSet()} est $O(1)$ et on réalise $n \cdot m$ créations d'ensembles, le coût total en mémoire est donc
$$O(n\cdot{m})$$

Par ailleurs, d'après la réponse~\ref{3.5} pour $n\cdot{m}$ créations d'ensembles, la complexité de l'algorithme est
$$O((n\cdot{m})\log(n\cdot{m}))$$
\subsubsection{}

\subsection{Bonus}
\subsubsection{}

\clearpage
\printbibliography

\end{document}