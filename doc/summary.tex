\documentclass[10pt,a4paper]{article}

\usepackage[a4paper]{geometry}
\usepackage[utf8]{inputenc}
\usepackage{fancyhdr}
\usepackage{titlesec}

\title{\textbf{TP Algorithmique \& Programmation} \\ Coloriage d'images}
\author{Vincent Bonnecuelle et Jibril Saffi}
\date{Octobre 2015}

\pagestyle{fancy}
\fancyhead[R]{1AA}
\fancyhead[L]{Grenoble INP \-- Ensimag}

\renewcommand\thesubsection{\arabic{subsection}}

\titleformat{\subsubsection}[runin]
{\normalfont\large\bfseries}{Question \thesubsubsection}{1em}{}

\titleformat{\subsection}[runin]
{\normalfont\large\bfseries}{\thesubsection}{1em}{}

\titleformat{\section}[runin]
{\normalfont\large\bfseries}{}{1em}{}

\renewcommand{\O}{\mathcal{O}}

\begin{document}
\maketitle
\hrule
\section*{Réponses aux questions}
\subsection{Lecture/écriture de données}
\subsubsection{} 
Nous avons identifié les cas d'erreurs suivants :
\begin{itemize}
\item le fichier n'est pas du bon type
\item il manque des valeurs dans le fichier par rapport au header spécifié
\item la forme du header n'est pas valide 
\item n*m ne renvoie pas un entier positif ou nul.
\end{itemize}
Les tests pour ces erreurs sont présents dans le fichier \textit{src/test/pixmap.cxx}.

\subsubsection{}
guarantee : Les erreurs concernant la validité des éléments de la structure sont impossibles car c++ est un langage fortement et statiquement typé or les couleurs sont modélisées avec des unsigned char.  
Nous avons identifié les cas d'erreurs suivants :
\begin{itemize}
\item le tableau bidimensionnel de pixels de couleur est vide
\item le tableau bidimensionnel de pixels de couleur n'est pas rectangulaire : il faut que les lignes aient la même longueur.
\end{itemize}
Les tests pour ces erreurs sont présents dans le fichier \textit{src/test/pixmap.cxx}.

\subsubsection{}
L'implémentation de la fonction \textbf{randomMonochrome} se trouve dans le fichier \textit{src/pixmap.cxx}.

\subsection{MakeSet() et FindSet()}

\subsubsection{}
Les tests suivants ont été réalisés pour MakeSet() et FindSet() :
\begin{itemize}
\item Initialisation avec une coordonnée
\item Création d'un ensemble et verification de l'existence de cet ensemble avec FindSet()
\end{itemize}

\subsubsection{}
Complexité de MakeSet = complexité de FindSet = $\Theta(1)$

\subsection{Union()}
\subsubsection{}
Soit 2 ensembles disjoints $S_{x}$ et $S_{y}$ représentés par les listes chainées $L_{x}$ et $L_{y}$. Afin de réaliser l'union de ces 2 ensembles, la solution naïve consiste à joindre la tête de $L_{y}$ à la queue de $L_{x}$ et de mettre à jour le representant de $L_{y}$ afin que tout les éléments de $L_{y}$ pointent sur le représentant de $L_{x}$.

TODO : faire un dessin 

Si l'on considère que le pire cas correspond au cas où chaque union ferait systématiquement parcourir l'ensemble le plus grand (pour mettre à jour le représentant). Ainsi pour $n$ unions on doit faire $n(n-1)?$ opérations\footnote{avec $k? = \sum\limits_{i=1}^{k}{i}$} or:
\[\O(n(n-1)?) =  \O(n(2(n-1))) = \O(n^2)\]

Aucune variable supplémentaire n'est nécessaire, on se contente de changer la structure de la liste donc le coût en mémoire est $\O(1)$.
\subsubsection{}

la création d'un ensemble est de complexité $\O(1)$ donc $n$ créations d'ensembles aura pour compexité $\O(n)$. Suivie d'unions successives, la complexité est de :
\[\O(n + (n-1)^2) = \O(n^2)\].

\subsubsection{}


\subsubsection{}
\textit{Voir fichier src/disjoint.cxx}

\subsubsection{}
\[\O(n\log(n))\]

\end{document}
